\pagestyle{plain}
\begin{center}
{\LARGE Abstract}\\[1cm]
\end{center}

The current diploma thesis addresses the problem of designing a complete and discreet navigational system for visually impaired people, supporting the traditional assisting methods like white cane and providing the user with the ability to move safer and faster throughout an urban environment, especially when it comes to crossing a road through a crosswalk. The ultimate goal is to implement an experimental setup that will be totally functional, will increase user's autonomy and will follow 3 main principles; portability, low intrusiveness and reliability. For satisfying the above criteria we made use of diverse image processing techniques and developed 3 different algorithms, for crosswalk detection, for pedestrian lights recognition and for obstacle detection, while also developing an Android application, which is responsible for the guidance and the interaction between the system and the user. The feedback to the user is provided through his smartphone, by using 4 different vibration patterns, because we wanted to utilize the haptic channel and limit the cognitive load connected with the acoustic channel. The final experimental setup includes the incorporation of a camera with a depth sensor, a microcomputer, a portable battery (power bank) and a smartphone.