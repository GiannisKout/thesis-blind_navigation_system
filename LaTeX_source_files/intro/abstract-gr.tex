\pagestyle{plain}
\begin{center}
{\LARGE Περίληψη}\\[1cm]
\end{center}

Η παρούσα διπλωματική εργασία έχει ως αντικείμενο τον σχεδιασμό ενός ολοκληρωμένου και διακριτικού συστήματος πλοήγησης για άτομα με προβλήματα όρασης, το οποίο θα λειτουργεί συνδυαστικά με παραδοσιακά βοηθήματα, όπως είναι το λευκό μπαστούνι, και θα παρέχει στον χρήστη την δυνατότητα ασφαλέστερης και γρηγορότερης μετακίνησης μέσα σε ένα αστικό περιβάλλον, εστιάζοντας περισσότερο στο πρόβλημα διάσχισης ενός δρόμου μέσω μιας διάβασης πεζών. Στόχος είναι η υλοποίηση μιας πειραματικής διάταξης η οποία θα είναι πλήρως λειτουργική, θα αυξάνει το βαθμό αυτονομίας του χρήστη και θα βασίζεται πάνω σε 3 βασικούς άξονες: την φορητότητα, την ελάχιστη επεμβατικότητα και την αξιοπιστία. Για την εξασφάλιση των παραπάνω χαρακτηριστικών χρησιμοποιήθηκαν τεχνικές επεξεργασίας εικόνας και αναπτύχθηκαν 3 διαφορετικοί αλγόριθμοι, που αφορούν την ανίχνευση διάβασης πεζών, την αναγνώριση της κατάστασης των φωτεινών σηματοδοτών και τον εντοπισμό εμποδίων, καθώς επίσης και μια εφαρμογή για Android smartphones, η οποία αναλαμβάνει την καθοδήγηση και αποτελεί την διεπιφάνεια αλληλεπίδρασης μεταξύ του συστήματος και του χρήστη. Η παροχή ανάδρασης στον χρήστη πραγματοποιείται μέσω του smartphone και τη χρήση 4 διαφορετικών μοτίβων δονήσεων, με σκοπό την αξιοποίηση του απτικού καναλιού και την αποσυμφόρηση του ακουστικού καναλιού του χρήστη. Η τελική πειραματική διάταξη περιλαμβάνει την ενσωμάτωση μιας κάμερας με αισθητήρα βάθους, ενός μικροϋπολογιστή, μιας φορητής μπαταρίας τύπου power bank και ένα smartphone.